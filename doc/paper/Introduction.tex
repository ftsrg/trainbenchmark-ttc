%!TEX root = ttc15-train-benchmark-sigma.tex

% \enlargethispage{10mm}

\section{Introduction}
\label{sec:Introduction}

%% Overview
In this paper we describe our solution for the \TTC Train Benchmark case study~\cite{Szarnyas2015} using the \SIGMA~\cite{Krikava2014}.
%% Scala
\SIGMA is a family of Scala\footnote{\url{http://scala-lang.org}} internal DSLs for model manipulation tasks such as model validation, model to model (M2M), and model to text (M2T) transformations.
Scala is a statically typed production-ready \emph{General-Purpose Language} (GPL) that supports both object-oriented and functional styles of programming.
It uses type inference to combine static type safety with a \emph{``look and feel''} close to dynamically typed languages.
Furthermore, it is supported by the major integrated development environments bringing EMF modeling to other IDEs than traditionally Eclipse (the solution was developed in IntelliJ IDEA\footnote{\url{https://www.jetbrains.com/idea/}}).

%% SIGMA
\SIGMA DSLs are embedded in Scala as a library allowing one to manipulate models using high-level constructs similar to the ones found in the external model manipulation DSLs.
The intent is to provide an approach that developers can use to implement many of the practical model manipulations within a familiar environment, with a reduced learning overhead as well as improved usability and performance.
The solution is based on the \emph{Eclipse Modeling Framework} (EMF)~\cite{EMF}, which is a popular meta-modeling framework widely used in both academia and industry, and which is directly supported by \SIGMA.

%% Organization
The complete source code is available on Github\footnote{\url{https://github.com/fikovnik/trainbenchmark-ttc}} in the fork of the original case study repository.
In the Appendix~\ref{sec:Install} we provide a steps how to install it locally.